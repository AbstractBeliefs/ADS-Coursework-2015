\documentclass[a4paper]{article}

\usepackage{listings}
\usepackage{graphicx}
\usepackage{fancyhdr}
\usepackage{color}
\usepackage{xcolor}
\usepackage[hidelinks]{hyperref}
\pagestyle{fancy}

\addtolength{\oddsidemargin}{-.875in}
\addtolength{\evensidemargin}{-.875in}
\addtolength{\headwidth}{1.75in}
\addtolength{\textwidth}{1.75in}

\lhead{Clarke-Wright Implementation}
\rhead{SET09117}
\lfoot{Report}
\cfoot{\thepage}
\rfoot{Gareth Pulham, 40099603}

\lstdefinestyle{customc}{
    belowcaptionskip=1\baselineskip,
    frame=single,
    breaklines=true,
    xleftmargin=\parindent,
    language=C,
    showstringspaces=false,
    basicstyle=\footnotesize\ttfamily,
    keywordstyle=\bfseries\color{green!40!black},
    commentstyle=\itshape\color{purple!40!black},
    identifierstyle=\color{blue},
    stringstyle=\color{orange},
}

\begin{document}
    \begin{titlepage}
        \title{Algorithms and Data Structures Coursework: An Implementation of the Clarke-Wright Route Optimisation in C}
        \author{Gareth Pulham, 40099603}
        \date{\today}
        \maketitle
        \thispagestyle{empty}
        \begin{abstract}
            As part of Napier University's Algorithms and Data Structures class (SET09117), students were tasked to implement and report upon an implementation
            of the Clarke-Wright \cite{CW} route optimisation algorithm. This document will cover one such implementation by student 40099603, covering how it was
            tested and how it performs in comparison to naive 1-route-per-customer routes.
        \end{abstract}
    \end{titlepage}

    \tableofcontents

    \section{Introduction}
    Vehicle routing is a classic algorithmic challenge, with a long history both theoretical and practical in nature.
    Identifying optimal routes has serious implications to the cost of business operations, and as such has been widely researched by delivery companies,
    such as UPS, Fedex, and others, because of this.
    In computer science, the best known form is the Travelling Salesman Problem, to which references date back at least as far as 1832. Computer science
    research into this field is particularly common in undergraduate studies as it offers a hands-on introduction to graph theory.

    One instance of research into the Vehicle Routing Problem field is the 1964 paper ``Scheduling of Vehicles from a Central Depot to a Number of Delivery Points''
    by Clarke and Wright \cite{CW}, which puts forward the so called ``Savings Method''. In this method, all delivery points are given a designated route of their
    own, the highest cost scenario for a multi-route delivery operation. Following this, all possible permutations of route merges are calculated and ranked
    biggest saving to worst. Given two delivery points $i$ and $j$, the saving that can be made is $ S_{ij} = c_{j0} + c_{0i} - c_{ij} $, as derived by
    Lysgaard \cite{Lysgaard} in ``Clarke and Wright's Savings Algorithm''. Once all these merge savings have been ranked, they are applied where possible. 
    Thanks to the ranking, the biggest saving route merges have the highest likelyhood of being applicable.

    \section{Method}
    As part of the algorithm implementation, the developer was mindful of the following test points:
    \begin{itemize}
        \item Correctness - the generated routes should cover all customers, and never exceed additional constraints such as delivery vehicle size.
        \item Speed and efficiency - the algorithm should run quickly and scale well with input size.
        \item Relative improvement - that is, the level of improvement over naive 1-route-per-customer routes.
    \end{itemize}
    These points were all tested in an automated and repeatable fashion across a wide dataset, ranging from 10 to 1000 customers.

        \subsection{Correctness}
        As part of the work, a tool, \texttt{Verify.jar} was provided that ensured that all customers in a given CSV file had been serviced, and that no
        routes generated ever exceeded the additional restraints. This tool additionally calculated the cost of the routes produced.

        \subsection{Speed and efficiency}
        The use of the algorithm in the test code was instrumented to measure the CPU time consumed between starting and finishing the algorithm,
        and the number of milliseconds consumed is printed during runtime.

        \subsection{Relative improvement}
        A stub version of the algorithm was also executed that produced naive 1-route-per-customer routes. These routes were also tested by \texttt{Verify.jar}
        and their costs recorded for comparison with the complete algorithm.
    

    \section{Results}
    blah blah blah

    \section{Conclusion}
    blah blah blah

    \bibliographystyle{abbrv}
    \bibliography{report}
    
    \section{Appendices}
        \subsection{Source code listings}
            This section contains the listings for functional code in this implementation of the Clarke-Wright algorithm.
            Complete versions of the entire project, including test data, build files, and the source for this report will be available from
            \url{https://github.com/AbstractBeliefs/ADS-Coursework-2015/} after the submission deadline (Saturday 21st November 2015, 23:55).
            \subsubsection{main.c}
                \lstinputlisting[style=customc]{../src/main.c}
            \subsubsection{ClarkeWright.h}
                \lstinputlisting[style=customc]{../src/ClarkeWright.h}
            \subsubsection{ClarkeWright.c}
                \lstinputlisting[style=customc]{../src/ClarkeWright.c}
        \subsection{Test code listings}
            \subsubsection{NaiveStub.c}
                \lstinputlisting[style=customc]{NaiveStub.c}
            \subsubsection{test\_costs.py}
                \lstinputlisting{../test_tools/test_costs.py}
            \subsubsection{test\_times.sh}
                \lstinputlisting{../test_tools/test_times.sh}
\end{document}
